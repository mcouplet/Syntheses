\documentclass[fr]{../../../eplnotes}
\usepackage{../../../eplunits}
\usepackage{pgfplots}
\usepackage{physics}

\pgfplotsset{compat=newest}
\usepgfplotslibrary{fillbetween}
\usetikzlibrary{external}
\usetikzlibrary{patterns}
\usetikzlibrary{arrows}
\usetikzlibrary{calc}
\tikzexternalize[prefix=figures/]
\numberwithin{equation}{section}

% Macros
\newcommand{\thermodiff}[3]{\left(\pdv{#1}{#2}\right)_{#3}} % Thermodynamic differential
\renewcommand{\d}[1]{\dif #1}
\newcommand{\Ru}{R_{\mathrm{u}}} % universal gas constant
\newcommand{\Wf}{W_{\mathrm{f}}} % terme dissipatif

\hypertitle{Thermodynamique et Énergétique}{5}{MECA}{1855}
{Adrien Couplet \and Mattéo Couplet}
{Miltiadis Papalexandris et Yann Bartosiewicz}[
\paragraph{Remarque des auteurs} 
Ce document a pour objectif de rassembler et de répondre en détails aux questions de théorie du cours en vue de l’examen.
Celles-ci ont soit été fournies par M. Papalexandris sur Moodle, soit proviennent d'anciens examens.
Le document suppose une compréhension préalable de la matière et ne fait office ni de synthèse, ni de syllabus.]

\section{Introduction et rappels}
\subsection{Est-ce que la chaleur échangée entre un système et son extérieur est une variable d'état du système ? Justifier votre réponse.}
Une \emph{variable d'état} est un paramètre qui caractérise un état et non une évolution entre deux états.
La chaleur échangée $Q$ entre un système et son extérieur décrit le passage d'un état à un autre et ne peux pas être mesurée pour un état thermodynamique donné.
Dans le cas d'une transformation (1-2) sans dissipation par frottements, on a par le premier principe
\[ Q_{12} = U_2 - U_1 + \int_1^2 p \dif v. \]
Dans le diagramme $(p,v)$ de la transformation, on peut passer de l'état 1 à l'état 2 par n'importe quelle courbe.
La valeur de $\int_1^2 p \dif v$ dépend de la courbe qu'on prend, d'où on conclut que $Q_{12}$ ne dépend pas de l'état initial et final mais bien du chemin parcouru pour passer de l'état initial à l'état final.

\subsection{Donner la définition d'un processus \textit{adiabatique}. 
Donner un exemple d'un processus adiabatique réversible ainsi qu'un exemple d'un processus adiabatique irréversible.}
Un processus est dit \emph{adiabatique} si lors de ce processus le système n'échange pas de chaleur avec son environnement ($Q = 0$). 
Cela n'implique pas pour autant que la température du système reste constante. 
Par le premier principe, seul un travail mécanique peut encore modifier l'état du système.

Une transformation est dite réversible si, partant d'un état donné, on pourrait la concevoir comme pouvant se produire indifféremment dans un sens ou dans l'autre.
Le concept de réversibilité est idéal : en pratique, tout processus est irréversible.
L'expression énergétique du second principe de la thermodynamique, aussi appelé principe d'évolution, s'écrit 
\[ Q + W_\mathrm{f} = \int_1^2T \dif S \]
où $\dif S$ est la différentielle de la fonction d'état entropie et $W_\mathrm{f}$ le terme dissipatif. 
Dans la réalité ce terme ne sera jamais nul. 
\begin{itemize}
	\item Dans le cas d'un processus \textbf{réversible} on a $W_\mathrm{f} = 0$. 
		Un processus adiabatique réversible est donc aussi isentropique car $\dif S = \delta Q/T = 0$. 
		Un exemple d'un tel processus se trouve dans le cycle de Carnot qui comprend une détente ainsi qu'une compression adiabatique réversible. 
		Il est à noter que le cycle de Carnot est un cycle théorique.
	\item Dans le cas d'un processus \textbf{irréversible} on a $W_\mathrm{f} > 0$.
		On retrouve une détente adiabatique irréversible dans les cycles frigorifiques. 
\end{itemize}

\subsection{Donner la définition mathématique ainsi que la signification physique des chaleurs massiques $c_p$ et $c_v$.}
Physiquement, la chaleur massique est l'action calorifique nécessaire pour obtenir un accroissement de température d'un degré Kelvin d'un kilo d'une substance considérée dans des conditions bien déterminées.
Dans les transformations à pression ou volume constants, on note les chaleurs massiques $c_p$ et $c_v$, respectivement.
Mathématiquement, la définition s'écrit
\[ c_p \eqdef \lim_{\Delta T \to 0} \frac{Q_p}{\Delta T}, \qquad c_v \eqdef \lim_{\Delta T \to 0} \frac{Q_v}{\Delta T} \]
où $Q_p$ et $Q_v$ désignent les actions calorifiques respectives nécessaires au relèvement $\Delta T$ de température à pression et à volume constant.

Par la définition, on peut écrire
\begin{equation*}
	\delta Q_v = c_v \dif T, \qquad \delta Q_p = c_p \dif T. 
\end{equation*}
Le premier principe nous donne la relation suivante
\begin{equation*} 
	\dif U + p \dif v = \dif H - v \dif p = \delta(Q + W_\mathrm{f}).
\end{equation*}
Dans le cas d'une transformation \emph{sans dissipation par frottements ($W_\mathrm{f} = 0$)} à volume constant, on a donc
\[ \dif U = c_v \dif T, \qquad c_v = \thermodiff{U}{T}{v} \]
et à pression constante,
\[ \dif H = c_p \dif T, \qquad c_p = \thermodiff{H}{T}{p} \]

\subsection{Donner la définition de l'énergie libre $F$ de Helmholtz ainsi que de l'enthalpie libre $G$ de Gibbs. 
Écrire l'équation de Gibbs sous la forme de différentielle de $F$ ainsi que de $G$.}
L'énergie libre de Helmholtz $F$ et l'enthalpie libre de Gibbs $G$ sont définies par
\begin{align*}
	F &\eqdef U - TS \\
	G &\eqdef H - TS.
\end{align*}
On trouve immédiatement leurs différentielles :
\begin{align*}
	\dif F &= \dif U - T \dif S - S \dif T \\
	\dif G &= \dif H - T \dif S - S \dif T.
\end{align*}
L'équation de Gibbs est la relation
\[ T \dif S = \dif U + p \dif v = \dif H - v \dif p \]
qui permet d'obtenir
\begin{align*}
	\dif F &= - p \dif v - S \dif T \\
	\dif G &= v \dif p - S \dif T.
\end{align*}

\iffalse % Interesting, but does not answer the question
\paragraph{L'énergie libre $F$ de Helmholtz} est une fonction d'état extensive dont la variation permet d'obtenir le travail utile susceptible d'être fourni par un système fermé, à température constante, au cours d'une transformation réversible. 
Considérons un système thermodynamique (fermé) évoluant d'un état 1 à un état 2, transformation que l'on suppose totalement réversible ($W_f = 0$) et à température constante $T$. Le premier principe de la thermodynamique s'exprime par :
\begin{equation} \Delta U + \Delta K + g\Delta z = Q + \underline{W_e} \end{equation}
où $\Delta U$ désigne la variation de l'énergie interne, $\Delta K$ la variation de l'énergie cinétique, $g\Delta z$ la variation de l'énergie potentielle, $Q$ l'échange de chaleur et $\underline{W_e}$ le travail des forces extérieures. On introduit $W = -W_e$ qui correspond au travail récupérable par le milieu extérieur et on suppose $\Delta K = g\Delta z = 0$. On obtient alors
\begin{equation} W = Q - \Delta U \label{eq:Q1ePrincipe}\end{equation}
Le 2\ieme principe de la thermodynamique s'exprime par:
\begin{equation} \Delta S = \frac{Q}{T} + W_f \qquad\Leftrightarrow\qquad Q = T\Delta S - TW_f \label{eq:Q2ePrincipe}\end{equation}
où $\Delta S$ est la variation d'entropie du système. Si on remplace \ref{eq:Q2ePrincipe} dans \ref{eq:Q1ePrincipe}, nous trouvons
\begin{equation} W = T\Delta S - TW_f - \Delta U = T(S_2 - S_1) - TW_f + U_1 - U_2 \end{equation}
Il est évident, d'après la relation obtenue, que $W$ sera maximum si la transformation est totalement réversible ($W_f = 0$) :
\begin{equation} W_\text{max} = (U_1 - TS_1)-(U_2-TS_2)\end{equation}
On définit alors 
\begin{equation} F = U - TS \end{equation} 
Ce qui nous donne au final 
\begin{equation} W_\text{max} = -\Delta F \qquad\text{où}\qquad \underbrace{\Delta F}_\text{travail utile} = \underbrace{\Delta U}_\text{énergie totale} - \underbrace{T\Delta S}_\text{énergie inutilisable}\end{equation}

\paragraph{Forme différentielle de $F$}
\begin{equation} \left.\begin{array}{ll} dF &= d(U-TS) = dU - TdS - SdT \\ dU &= -pdV + TdS \end{array}\right\} \Rightarrow dF = -pdV - SdT\end{equation}

\paragraph{L'enthalpie libre $G$ de Gibbs} est une fonction d'état extensive dont la variation permet d'obtenir l'enthalpie utile susceptible d'être fournie par un système fermé, à température et à pression constante, au cours d'une transformation réversible. On considère le même système que précedemment avec la condition supplémentaire que la pression est constante. Par la définition de l'enthalpie :
\begin{equation} H \triangleq U + pV\end{equation}
De la même manière que pour l'énergie libre de Helmholtz, on trouve :
\begin{equation} W = Q - \Delta H \label{H1ePrincipe}\end{equation}
où $W$ correspond à nouveau au travail récupérable par le milieu extérieur. Si on remplace \ref{eq:Q2ePrincipe} dans \ref{H1ePrincipe} et qu'on prend on compte le fait que le processus est réversible, on obtient :
\begin{equation} W_\text{max} = (H_1 - TS_1) - (H_2 - TS_2) \end{equation}
On définit alors
\begin{equation} G = H - TS \end{equation}

\paragraph{Forme différentielle de G}
\begin{equation} \left.\begin{array}{ll} dG &= d(H-TS) = dH - TdS - SdT \\ dH &= Vdp + TdS \end{array}\right\} \Rightarrow dG = Vdp - SdT \label{eq:gibbs-duhem}\end{equation}

\fi

\subsection{Dériver l'équation $\alpha = p\beta \kappa$.}
Les variables d'état $p$, $v$ et $T$ d'un fluide sont liées par une équation d'état, que l'on peut écrire sous forme du système différentiel :
\[
	\begin{pmatrix} \dif p \\ \dif v \\ \dif T \end{pmatrix} = 
	\begin{pmatrix}
		0 & \thermodiff{p}{v}{T} & \thermodiff{p}{T}{v} \\
		\thermodiff{v}{p}{T} & 0 & \thermodiff{v}{T}{p} \\
		\thermodiff{T}{p}{v} & \thermodiff{T}{v}{p} & 0
	\end{pmatrix}
	\begin{pmatrix} \dif p \\ \dif v \\ \dif T \end{pmatrix} 
\]
On définit les coefficients suivants :
\begin{itemize}
	\item le coefficient de dilatation isobare
		$ \alpha \eqdef \frac{1}{v} \thermodiff{v}{T}{p} $ ;
	\item le coefficient de dilatation isochore
		$ \beta \eqdef \frac{1}{p} \thermodiff{p}{T}{v} $ ;
	\item le coefficient de compressibilité isotherme
		$ \kappa \eqdef -\frac{1}{v} \thermodiff{v}{p}{T} $.
\end{itemize}
Grâce à ces coefficients, on peut réécrire le système :
\[
	\begin{pmatrix} \dif p \\ \dif v \\ \dif T \end{pmatrix} = 
	\begin{pmatrix}
		0 & -\frac{1}{\kappa v} & \beta p \\
		-\kappa v & 0 & \alpha v \\
		\frac{1}{\beta p} & \frac{1}{\alpha v} & 0
	\end{pmatrix}
	\begin{pmatrix} \dif p \\ \dif v \\ \dif T \end{pmatrix} 
\]
On trouve l'équation recherchée à l'aide des variations de deux des trois variables d'état: 
\begin{align*}
	\dif p &= -\frac{1}{\kappa v} \dif v + \beta p \dif T \\
	\dif v &= -\kappa v \dif p + \alpha v \dif T
\end{align*}
En remplaçant l'expression de $\dif v$ dans l'expression de $\dif p$, on obtient
\[ \dif p = \dif p - \frac{\alpha}{\kappa} \dif T + \beta p \dif T \]
d'où on tire $ \alpha = p \beta \kappa $.
L'équation d'état d'une substance est donc complètement déterminée par la connaissance de deux des trois coefficients $\alpha$, $\beta$ ou $\kappa$.

\subsection{Démontrer que les relations suivantes sont valables pour toutes les espèces.}
\[ \thermodiff{T}{p}{S} = \thermodiff{v}{S}{p}, \qquad
	\thermodiff{S}{p}{T} = -\thermodiff{v}{T}{p} \]

On part de l'équation de Gibbs sous la forme de différentielle de $H$ et $G$ :
\begin{align*}
	\dif H &= T \dif S + v \dif p \\
	\dif G &= v \dif p - S \dif T
\end{align*}
qui nous donne les dérivées partielles suivantes
\[ \begin{matrix}
	\thermodiff{H}{S}{p} = T & \thermodiff{H}{p}{S} = v \\
	\thermodiff{G}{p}{T} = v & \thermodiff{G}{T}{p} = -S
\end{matrix} \]
Par le théorème de Schwarz
\footnote{Si $f$ est deux fois dérivable en un point, alors sa matrice hessienne y est symétrique}
appliqué aux dérivées partielles de $H$ et $G$ :
\begin{align*}
	\pdv{H}{S}{p} &= \pdv{H}{p}{S} \\
	\pdv{G}{T}{p} &= \pdv{G}{p}{T}
\end{align*}
on trouve les relations recherchées :
\begin{align}
	%\thermodiff{T}{v}{S} &= -\thermodiff{p}{S}{v} \\
	\thermodiff{T}{p}{S} &= \thermodiff{v}{S}{p} \notag \\
	\thermodiff{v}{T}{p} &= -\thermodiff{S}{p}{T}.
\end{align}

\subsection{Dériver l'équation $c_p-c_v = \alpha\beta pvT$.\label{q:1_7}}
On part de l'équation de Gibbs
\[ T \dif S = \dif U + p \dif v \]
La différentielle de $U$ peut s'écrire de trois manières différentes :
\[
	\begin{pmatrix} \dif U \\ \dif U \\ \dif U \end{pmatrix} = 
	\begin{pmatrix}
		0 & \thermodiff{U}{v}{T} & \thermodiff{U}{T}{v} \\
		\thermodiff{U}{p}{T} & 0 & \thermodiff{U}{T}{p} \\
		\thermodiff{U}{p}{v} & \thermodiff{U}{v}{p} & 0
	\end{pmatrix}
	\begin{pmatrix} \dif p \\ \dif v \\ \dif T \end{pmatrix}
\]
On peut donc exprimer $T\dif S$ de trois manières différentes :
\[ 
	\begin{pmatrix} T \dif S \\ T \dif S \\ T \dif S \end{pmatrix} = 
	\begin{pmatrix}
		0 & \thermodiff{U}{v}{T} + p & \thermodiff{U}{T}{v} \\
		\thermodiff{U}{p}{T} & p & \thermodiff{U}{T}{p} \\
		\thermodiff{U}{p}{v} & \thermodiff{U}{v}{p} + p & 0
	\end{pmatrix}
	\begin{pmatrix} \dif p \\ \dif v \\ \dif T \end{pmatrix}
\]
On annule le second terme de la diagonale avec $ \dif v = \thermodiff{v}{p}{T} \dif p + \thermodiff{v}{T}{p} \dif T $ :
\[ 
	\begin{pmatrix} T \dif S \\ T \dif S \\ T \dif S \end{pmatrix} = 
	\begin{pmatrix}
		0 & \thermodiff{U}{v}{T} + p & \thermodiff{U}{T}{v} \\
		\thermodiff{U}{p}{T} + p \thermodiff{v}{p}{T} & 0 & \thermodiff{U}{T}{p} + p \thermodiff{v}{T}{p} \\
		\thermodiff{U}{p}{v} & \thermodiff{U}{v}{p} + p & 0
	\end{pmatrix}
	\begin{pmatrix} \dif p \\ \dif v \\ \dif T \end{pmatrix}
\]
Par introduction de nouveaux coefficients calorifiques, on note
\[ 
	\begin{pmatrix} T \dif S \\ T \dif S \\ T \dif S \end{pmatrix} = 
	\begin{pmatrix}
		0	& l_T	& c_v	\\
		h_T	& 0		& c_p	\\
		h_v	& l_p	& 0
	\end{pmatrix}
	\begin{pmatrix} \dif p \\ \dif v \\ \dif T \end{pmatrix}
\]
où on définit ces coefficients :
\begin{itemize}
	\item coefficient calorifique de dilatation à température constante :
		\[ l_T = \thermodiff{U}{v}{T} + p = \thermodiff{H}{v}{T} - v \thermodiff{p}{v}{T} \]
	\item coefficient calorifique de compressbilité à température constante :
		\[ h_T = \thermodiff{U}{p}{T} + p \thermodiff{v}{p}{T} = \thermodiff{H}{p}{T} - v \]
	\item coefficient calorifique de compressibilité sous volume constant :
		\[ h_v = \thermodiff{U}{p}{v} = \thermodiff{H}{p}{v} - v \]
	\item coefficient calorifique de dilatation sous pression constante :
		\[ l_p = \thermodiff{U}{v}{p} + p = \thermodiff{H}{v}{p} \]
\end{itemize}
On annule la dernière colonne avec $ \dif T = \thermodiff{T}{p}{v} \dif p + \thermodiff{T}{v}{p} \dif v = \frac{1}{\beta p} \dif p + \frac{1}{\alpha v} \dif T $ :
\[ 
	\begin{pmatrix} T \dif S \\ T \dif S \\ T \dif S \end{pmatrix} = 
	\begin{pmatrix}
		\frac{c_v}{\beta p}			& l_T + \frac{c_v}{\alpha v}	& 0	\\
		h_T + \frac{c_p}{\beta p}	& \frac{c_p}{\alpha v}			& 0	\\
		h_v							& l_p							& 0
	\end{pmatrix}
	\begin{pmatrix} \dif p \\ \dif v \\ \dif T \end{pmatrix}
\]
Par identification des coefficients, on peut exprimer les coefficients $l_T$, $h_T$, $h_v$ et $l_p$ en fonction des chaleurs massiques $c_p$ et $c_v$ et des coefficients de dilatation $\alpha$ et $\beta$ :
\[
	l_T = \frac{c_p-c_v}{\alpha v} \qquad
	h_T = \frac{c_v-c_p}{\beta p} \qquad
	h_v = \frac{c_v}{\beta p} \qquad
	l_p = \frac{c_p}{\alpha v}
\]
On peut alors exprimer les dérivées partielles de l'entropie en fonction de $\alpha$, $\beta$, $c_p$ et $c_v$ uniquement :
\[
	\begin{pmatrix}
		0						& \thermodiff{S}{v}{T}	& \thermodiff{S}{T}{v}	\\
		\thermodiff{S}{p}{T}	& 0						& \thermodiff{S}{T}{p}	\\
		\thermodiff{S}{p}{v}	& \thermodiff{S}{v}{p}	& 0
	\end{pmatrix}
	=
	\begin{pmatrix}
		0							& \frac{c_p-c_v}{\alpha v T}	& \frac{c_v}{T}	\\
		\frac{c_v-c_p}{\beta p T}	& 0								& \frac{c_p}{T}	\\
		\frac{c_v}{\beta p T}		& \frac{c_p}{\alpha v T}		& 0
	\end{pmatrix}
\]
Le théorème de Schwarz appliqué aux dérivées partielles de $G$ nous donne la relation
\[ \thermodiff{v}{T}{p} = - \thermodiff{S}{p}{T} \]
En exprimant ces différentielles par les coefficient calorifiques, on trouve
\begin{align*}
	\alpha v &= \frac{c_p - c_v}{\beta p T} \\
	c_p - c_v &= \alpha \beta p v T
\end{align*}
Il résulte de ces considérations que les fonctions d'état peuvent se calculer si on connait deux des trois coefficients physiques $\alpha$, $\beta$, $\kappa$ et la chaleur massique $c_p$ (ou $c_v$).

\subsection{Dériver l'équation $l_T = \frac{c_p-c_v}{\alpha v}$.}
Le développement pour cette question est un sous-ensemble de la réponse à la question précédente.


\section{Gaz idéaux}
\subsection{Démontrer que l'entropie d'un gaz idéal est donnée par l'expression suivante}
\[ s = s_0(T_0) - R_g \ln\frac{p}{p_0} + \int_{T_0}^T c_p \frac{\dif T}{T}
	=  s_0(T_0) + R_g \ln\frac{\rho_0}{\rho} + \int_{T_0}^T c_v \frac{\dif T}{T}
\]

La forme particulière de l'équation d'état du gaz idéal $pv = RT$ permet d'écrire les égalités suivantes :
\begin{align*}
	\alpha &\eqdef \frac{1}{v}\thermodiff{v}{T}{p} = \frac{p}{RT}\frac{R}{p} = \frac{1}{T}; \\
	\beta  &\eqdef \frac{1}{p}\thermodiff{p}{T}{v} = \frac{v}{RT}\frac{R}{v} = \frac{1}{T},
\end{align*}
dont on déduit la forme simple de la relation reliant les chaleurs massiques :
\[ c_p - c_v = \alpha\beta p v T = R. \]
On peut reprendre les expressions de la différentielle de l'entropie qui peuvent s'écrire
comme suit:
\[
	\begin{pmatrix} \dif S \\ \dif S \\ \dif S \end{pmatrix} 
	= 
	\begin{pmatrix}
		0 & \beta p & \frac{c_v}{T} \\
		-\alpha v & 0 & \frac{c_p}{T} \\
		\frac{c_v}{\beta p T} & \frac{c_p}{\alpha v T} & 0
	\end{pmatrix}
	\begin{pmatrix} \dif p \\ \dif v \\ \dif T \end{pmatrix}
	=
	\begin{pmatrix}
		0 & \frac{R}{v} & \frac{c_v}{T} \\
		-\frac{R}{p} & 0 & \frac{c_p}{T} \\
		\frac{c_v}{p} & \frac{c_p}{v} & 0
	\end{pmatrix}
	\begin{pmatrix} \dif p \\ \dif v \\ \dif T \end{pmatrix}
\]
La première équation de ce système nous donne l'expression
\[ \dif S = R\frac{\dif v}{v} + c_v\frac{\dif T}{T} \]
qu'on intègre de $T_0$ à $T$ pour obtenir
\[ s = s_0(T_0) + R\ln\frac{v}{v_0} + \int_{T_0}^T c_v\frac{\dif T}{T}
	 = s_0(T_0) + R\ln\frac{\rho_0}{\rho} + \int_{T_0}^T c_v\frac{\dif T}{T}. \]
La seconde équation nous donne l'expression
\[ \dif S = -R\frac{\dif p}{p} + c_p \frac{\dif T}{T} \]
qu'on intègre de $T_0$ à $T$ pour obtenir
\[ s = s_0(T_0) - R\ln\frac{p}{p_0} + \int_{T_0}^T c_p\frac{\dif T}{T}. \]

\subsection{Donner l'expression de variation d'entropie
quand on chauffe un gaz idéal de $T_i$ jusqu'à $T_f$ sous pression constante.
Donner également l'expression de variation d'entropie quand l'élévation de température
est effectuée sous volume constant.
Lequel des deux processus donne la plus grande augmentation d'entropie ?\label{q:2_2}}
À pression constante :
\[ \Delta S_p = \int_{T_i}^{T_f} c_p \frac{\dif T}{T}. \]
À volume constant :
\[ \Delta S_v = \int_{T_i}^{T_f} c_v \frac{\dif T}{T} 
= \int_{T_i}^{T_f} (c_p-R) \frac{\dif T}{T} = \Delta S_p - R\ln\frac{T_f}{T_i} \]
Si $T_f > T_i$, on voit que le processus à pression constante produira une plus grande
variation d'entropie que le processus à volume constant.

\iffalse
	On reprend la matrice $S'$ de \ref{eq:matriceSideal}. À pression constante la première équation du système devient 
	\begin{equation} dS = -R\frac{dp}{p} + c_p\frac{dT}{T} = c_p\frac{dT}{T} \end{equation}
	Sous l'hypothèse que le gaz considéré est un gaz parfait ($c_p$ constant), cette relation s'intègre immédiatement pour obtenir l'expression de variation d'entropie quand on chauffe un gaz idéal de $T_i$ jusqu'à $T_f$ sous pression constante :
	\begin{equation} \Delta S = c_p\ln{\frac{T_f}{T_i}} \qquad \text{ou} \qquad \frac{T_f}{T_i} = \exp\left(\frac{\Delta S}{c_p}\right) \label{eq:isobare_ideal}\end{equation}
	À volume constant la deuxième équation du système devient
	\begin{equation} dS = R\frac{dv}{v} + c_v\frac{dT}{T} = c_v\frac{dT}{T} \end{equation}
	Sous l'hypothèse que le gaz considéré est un gaz parfait ($c_v$ constant), cette relation s'intègre immédiatement pour obtenir l'expression de variation d'entropie quand on chauffe un gaz idéal de $T_i$ jusqu'à $T_f$ sous volume constant :
	\begin{equation} \Delta S = c_v\ln{\frac{T_f}{T_i}} \qquad \text{ou} \qquad \frac{T_f}{T_i} = \exp\left(\frac{\Delta S}{c_v}\right) \label{eq:isochore_ideal}\end{equation}
	Le diagramme ($T$,$S$) (Figure \ref{fig:TS_gaz_parfait}) est constitué du réseau des isobares et des isochores, réseau obtenu par translation horizontale d'une isobare et d'une isochore arbitraire, soit celles passant par l'origine conventionnelle ($T_0 = \SI{0}{\kelvin}$, $S_0 = \SI{0}{\kilo\joule\per\kilo\gram\per\kelvin}$). On observe sur le diagramme que les isochores ont une pente plus élevée que les isobares, car $c_p > c_v$. À conditions initiales égales \textbf{le processus isobare donnera une plus grande augmentation d'entropie que le processus isochore}. 
	On ne manquera pas de noter que la levée des approximations $c_v = C^\text{te}$ et $c_p = C^\text{te}$ utilisées pour intégrer facilement l'expression de la variation d'entropie n'affecte l'allure générale des courbes isochores et isobares que dans la mesure de la variation relative des chaleurs massiques, et cette variation reste modérée pour des intervalles restreints de température.
	\begin{figure}[p]\centering
		\tikzsetnextfilename{TS_gaz_parfait}
		\begin{tikzpicture}
			\begin{axis}[enlargelimits=true,grid=major,ylabel=$T\;(\si{\kelvin})$,xlabel=$S\;(\si{\joule\per\kilo\gram\per\kelvin})$,xticklabels={,,},yticklabels={,,},extra y ticks=273.15,
		extra y tick style={grid=major, grid style={black}},
		extra y tick labels={$T_0$},
		extra x ticks=159,
		extra x tick style={grid=major, grid style={black}},
		extra x tick labels={$S_0$}]
				\addplot [blue,domain=253.15:373.15,samples=200] ({159 + 1004*ln(x/273.15)},x);
				\addplot [red,domain=253.15:373.15,samples=200] ({159 + 405*ln(x/273.15)},x);
				\addplot [blue,dashed,domain=253.15:373.15,samples=200] ({200 + 1004*ln(x/273.15)},x);
				\addplot [red,dashed,domain=253.15:373.15,samples=200] ({100 + 405*ln(x/273.15)},x);
				\legend{$dp = 0$,$dv = 0$}
			\end{axis}
		\end{tikzpicture}
		\caption{Diagramme ($T$,$S$) d'un gaz parfait}
		\label{fig:TS_gaz_parfait}
	\end{figure}
\fi

\subsection{Définir la transformation polytropique et
en établir les équations appliquées au gaz idéal}
On appelle transformation polytropique une transformation
dans laquelle est respectée l'une des relations :
\[ \frac{\dif H}{T \dif S} = \Psi = C^\text{te}
\qquad \frac{\dif U}{T \dif S} = \Phi = C^\text{te} \]
On notera que le rapport entre les coefficients $\Psi$ et $\Phi$ est constant
et que le respect d'une des relations entraîne nécessairement celui de l'autre
puisque l'on peut écrire :
\[ \frac{\Psi}{\Phi} = \frac{\dif H}{\dif U} = \frac{c_p}{c_v} = \gamma \]
où le quotient $\gamma$ est le \emph{coefficient de Poisson}, 
propriété intrinsèque du gaz considéré.

Dans le plan $(T,S)$ du gaz idéal, l'équation de la transformation polytropique
se déduit de sa définition même en y explicitant $\dif U$ ou $\dif H$ :
\[ \dif U = c_v \dif T = \Phi T \dif S
\qquad \text{ou} \qquad \dif H = c_p \dif T = \Psi T \dif S \]
La séparation des variables de cette équation donne :
\[ \dif S = \frac{c_v}{\Phi}\frac{\dif T}{T}
= \frac{c_p}{\Psi}\frac{\dif T}{T}
= c\frac{\dif T}{T} \]
avec $c$ une chaleur massique propre à la transformation.

Dans le plan $(p,v)$, l'équation de la transformation polytropique s'obtient
en explicitant les expressions de $v \dif p$ et $p \dif v$ sous la forme :
\begin{align*}
	v\d{p} &= \d{H} - T\d{S} = (\Psi-1)T\d{S} \\
	p\d{v} &= T\d{S} - \d{U} = (1-\Phi)T\d{S}
\end{align*}
ce qui donne par division membre à membre
\[ \frac{\d{p}/p}{\d{v}/v} = \frac{1-\Psi}{\Phi-1} \eqdef -m \]
Dans un intervalle où on peut admettre la constance des chaleurs massiques, et donc celles de $\Phi$, $\Psi$ et $m$, l'équation ci-dessus s'intègre sous la forme :
\begin{equation} pv^m = C^\text{te}\end{equation}
ce qui peut encore s'écrire :
\begin{equation} \frac{p_2}{p_1} = \left(\frac{v_1}{v_2}\right)^{m} \qquad \frac{T_2}{T_1} = \left(\frac{v_1}{v_2}\right)^{m-1} \qquad \frac{T_2}{T_1} = \left(\frac{p_2}{p_1}\right)^{\frac{m-1}{m}} \end{equation}
Par ailleurs, l'utilisation du coefficient de Poisson $\gamma$ dans la relation liant l'exposant $m$ à $\Phi$ et $\Psi$ permet d'expliciter séparément ces deux paramètres en fonction de $m$ :
\begin{equation} \Phi = \frac{m-1}{m-\gamma} \qquad \Psi = \gamma\frac{m-1}{m-\gamma} \end{equation}

\subsection{Quelle est la loi de Dalton ?
Donner l'équation d'état d'un mélange homogène des gaz idéaux en appliquant cette loi.}
Loi de Dalton sur l'additivité des pressions partielles:
\begin{center}
	\textit{Quand on mélange plusieurs gaz qui ne réagissent pas chimiquement,
	chacun d'eux se répartit uniformément dans tout le volume offert comme s'il était seul
	et la pression du mélange a pour valeur la somme des pressions dites partielles
	qu'aurait chacun d'eux s'il occupait seul le volume total du mélange.}
\end{center}
On peut dès lors écrire, pour un volume total $V$ de mélange
\[ pV = \sum_i p_i V = \sum_i n_i \Ru T \]
ce qui, si on l'identifie à la relation
\[ pV = \sum_i M_i R T \]
donne immédiatement la constante massique $R$ du mélange
\[ R = \frac{\sum_i n_i}{\sum_i M_i} = \frac{\Ru}{M_m} \]
où $M_m$ est la masse molaire moyenne du mélange.

\iffalse
La \textbf{loi de Dalton} stipule que quand on mélange plusieurs gaz qui ne réagissent pas chimiquement, chacun d'eux se répartit uniformément dans tout le volume offert comme s'il était seul et la pression du mélange a pour valeur la sommes des pressions dites partielles qu'aurait chacun d'eux s'il occupait seul le volume total du mélange.
\begin{equation} p = \sum p_i \end{equation}
On peut dès lors écrire, pour un volume total $V$ de mélange :
\begin{equation} pV = \sum p_iV = \sum M_iR_iT = \sum M_i\frac{R_u}{M_{mi}}T = \sum n_iR_uT \end{equation}
On en déduit ensuite les fonctions d'états. L'enthalpie et l'énergie interne ne dépendant que de la température, et celle-ci étant la même pour tout les constituants dans le mélange, l'enthalpie et l'énergie interne de celui-ci est la somme des enthalpies des constituants pris à la même température. On peut donc écrire $h$ et $u$ désignant respectivement l'enthalpie et l'énergie interne par \si{\kilo\gram} de mélange :
\begin{align} h &= \int_0^tc_pdt = \frac{1}{M_m}\int_0^tC_pdt = \frac{1}{M_m}\sum_i\int_0^tC_{pi}dt \\ u &= \int_0^tc_vdt = \frac{1}{M_m}\int_0^tC_vdt = \frac{1}{M_m}\sum_i\int_0^tC_{vi}dt \end{align}
L'entropie d'un mélange de gaz idéaux mérite une attention particulière, car elle dépend à la fois de la température et de la pression. Avant mélange (état 1), les constituants d'un système occupent aux mêmes conditions de pression et de température des volumes séparés. L'entropie du système est alors la somme des entropies individuelles pondérées par les fractions définissant le mélange, soit, en notant l'entropie molaire $S_m$ :
\begin{equation} S_{m,1} = \sum_iS_{mi}(p,t) \end{equation}
Après mélange (état 2), chacun des constituants s'étant détendu de façon isotherme dans tout le volume disponible, de la pression totale $p$ à la pression partielle $p_i$, l'entropie prend pour valeur :
\begin{equation} S_{m,2} = \sum_iS_{mi}(p_i,t) \end{equation}
L'opération de mélange a donc entraîné l'accroissement d'entropie molaire :
\begin{equation} \Delta S_m = \sum_i\left(S_{mi}(p_i,t) - S_{mi}(p,t)\right) = -\sum_iR_u\ln{\frac{p_i}{p}} \end{equation}
ce qui peut encore s'écrire en termes d'entropie massique :
\begin{equation} \Delta S = \frac{\Delta S_m}{M_m} = -R\sum_i\ln{\frac{p_i}{p}} \end{equation}
\fi

\subsection{Démontrer que si l'énergie interne et l'enthalpie d'une substance
ne dépendent que de la température, alors cette substance est un gaz idéal.}
Puisque l'énergie interne et l'enthalpie ne dépendent pas de la pression ni de la masse volumique,
on peut écrire
\begin{align*}
	\dd{U} &= \thermodiff{U}{v}{T}\dd{v} + \thermodiff{U}{T}{v}\dd{T} = c_v\dd{T} \\
	\dd{H} &= \thermodiff{H}{p}{T}\dd{p} + \thermodiff{H}{T}{p}\dd{T} = c_p\dd{T}
\end{align*}
où $c_v$ et $c_p$ ne dépendent que de la température.
La relation de Gibbs devient alors
\[ \dd{S} = \frac{c_v}{T}\dd{T} + \frac{p}{T}\dd{v} = \frac{c_p}{T}\dd{T} - \frac{v}{T}\dd{p}. \]
L'application du théorème de Schwarz aux dérivées partielles apparaissant
dans ces expressions différentielles donne lieu aux égalités suivantes:
\begin{align*}
	0 = \pdv{v}\left(\frac{c_v}{T}\right)_T &= \pdv{T}\left(\frac{p}{T}\right)_v
	= \frac{1}{T}\thermodiff{p}{T}{v} - \frac{p}{T^2}
	\\
	0 = \pdv{p}\left(\frac{c_p}{T}\right)_T &= \pdv{T}\left(-\frac{v}{T}\right)_p
	= -\frac{1}{T}\thermodiff{v}{T}{p} + \frac{v}{T^2}
\end{align*}
ce qui nous donne, par définition de $\alpha$ et $\beta$
\begin{align*}
	\frac{1}{v}\thermodiff{v}{T}{p} &= \alpha = \frac{1}{T} \\
	\frac{1}{p}\thermodiff{p}{T}{v} &= \beta  = \frac{1}{T}
\end{align*}
ce qui prouve que le gaz en question est bien un gaz idéal.
% TODO: comment retomber sur pv=RT à partir d'ici?

\iffalse
% proof that for ideal gas, c_p and c_v do not depend on p and v
On reprend les expressions de la différentielle de l'entropie qui peuvent s'écrire comme suit
pour un gaz idéal:
\[ \d{S} = \frac{R}{v}\d{v} + \frac{c_v}{T}\d{T} = -\frac{R}{p}\d{p} + \frac{c_p}{T}\d{T}
= \frac{c_v}{p}\d{p} + \frac{c_p}{v}\d{v} \]
L'application du théorème de Schwarz aux dérivées partielles apparaissant dans ces expressions
différentielles donne lieu aux égalités suivantes:
\begin{align*}
	\pdv{S}{v}{T} &= \frac{1}{T}\pdv{c_v}{v} = 
	\pdv{S}{T}{v} = \pdv{T}\left(\frac{R}{v}\right) = 0 \\
	\pdv{S}{p}{T} &= \frac{1}{T}\pdv{c_p}{p} = 
	\pdv{S}{T}{p} = \pdv{T}\left(-\frac{R}{p}\right) = 0 \\
\end{align*}
Pour l'état gazeux idéal, les chaleurs massiques $c_p$ et $c_v$ sont donc indépendantes
du volume massique et de la pression.
\fi

\iffalse
Un gaz est dit idéal si il suit l'équation $pv = RT$. Pour l'état gazeux idéal, les chaleurs massiques $c_v$ et $c_p$ sont indépendantes du volume massique et de la pression. Il en va donc de même pour l'énergie interne $U$ et l'enthalpie $H$, puisque l'on a, en prenant l'expression de  $dS$ dans la matrice \ref{eq:matriceSideal} :
\begin{align} dU &= TdS - pdv = \frac{RT}{v}dv + c_vdT - pdv = c_vdT \\ dH &= TdS + vdp = \frac{-RT}{p}dp + c_pdT + vdp = c_pdT \end{align}
Cette propriété est connue sous le nom de \textbf{loi de Joule du gaz idéal}. Ces considérations permettent de calculer les variations de l'énergie interne et de l'enthalpie pour toute transformation du gaz idéal au moyen des relations :
\begin{align} \Delta U &= \int_{T_1}^{T_2}c_vdT \\ \Delta H &= \int_{T_1}^{T_2}c_pdT \end{align}
\fi

\subsection{Considérer l'équation des transformations isochores
ainsi que l'équation des transformations isobares d'un gaz idéal.
Laquelle des deux équations a la pente la plus élevée dans le plan $(T,S)$?
Justifier votre réponse.}
Par l'équation de Gibbs pour les gaz idéaux, on a
\begin{align*}
	\text{à volume constant}\quad & \dd{S} = c_v\frac{\dd{T}}{T} \quad
	\Rightarrow \quad \dv{T}{S} = \frac{T}{c_v} \\
	\text{à pression constante}\quad & \dd{S} = c_p\frac{\dd{T}}{T} \quad
	\Rightarrow \quad \dv{T}{S} = \frac{T}{c_p}
\end{align*}
Les isochores ont dont une pente plus élevée que les isobares, car $c_p > c_v$.

\iffalse
On reprend les équations \ref{eq:isobare_ideal} et \ref{eq:isochore_ideal} de la question \ref{q:2_2}:
\begin{align} p = C^\text{te} \qquad &\Rightarrow \qquad \frac{T_f}{T_i} = \exp\left(\frac{\Delta S}{c_p}\right) \\ v = C^\text{te}  \qquad &\Rightarrow \qquad \frac{T_f}{T_i} = \exp\left(\frac{\Delta S}{c_v}\right) \end{align}
Comme $c_p > c_v$, une transformations isochore aura une pente plus élevée dans le diagramme ($T$,$S$) qu'une transformation isobare sous les mêmes conditions initiales. Ce résultat est également visible sur le diagramme à la figure \ref{fig:TS_gaz_parfait}.
\fi

\subsection{Démontrer que l'expression suivante est valable pour des gaz idéaux $c_p-c_v = R$.}
On sait que les chaleurs massiques d'un gaz quelconque sont liées par la relation 
\[ c_p-c_v = \alpha\beta pvT \]
Si on reprend les définitions de $\alpha$ et de $\beta$ dans le cas des gaz idéaux ($pv=RT$),
\[ \alpha \eqdef \frac{1}{v}\thermodiff{v}{T}{p} = \frac{R}{pv} = \frac{1}{T}
\qquad \beta \eqdef \frac{1}{p}\thermodiff{p}{T}{v} = \frac{R}{pv} = \frac{1}{T} \]
on trouve la relation liant les chaleurs massiques $c_p$ et $c_v$ :
\begin{equation} c_p-c_v = \frac{pv}{T} = R \end{equation}

\subsection{Démontrer que le travail produit par l'expansion isotherme de $N$ moles d'un gaz idéal
est donnée par $ W = -N\Ru T\ln(V_f/V_i) $, $V_i$ étant le volume initial
et $V_f$ le volume final du gaz.}
Si on suppose la transformation réversible ($\Wf=0$), le travail produit a pour expression
\[ W = \int -p\dd{V} \]
Par l'équation d'état des gaz idéaux $pV = N \Ru T$ et sachant que la température est constante,
\[ W = -N\Ru T\int_{V_i}^{V_f} \frac{\dd{V}}{V} = -N\Ru T\ln\left(\frac{V_f}{V_i}\right) \]
Notons que ce travail est négatif si $V_f > V_i$ (le gaz fournit un travail).

\iffalse
\begin{equation} W = -NR_uT\ln\left(\frac{V_f}{V_i}\right) \label{eq:q2_8}\end{equation}
où $V_i$ est le volume inital et $V_f$ le volume final de gaz.
Prenons pour exemple un piston sur lequel une pression est excercée par le milieu extérieur :
\begin{equation} \delta W = F_\text{ext}dx = P_\text{ext}Sdx = P_\text{ext}dV \qquad \text{avec} \qquad (dV>0)\end{equation} 
Le travail est fourni par le système contre $F_\text{ext}$, donc $\delta W$ doit être négatif. On a donc 
\begin{equation} \delta W = -P_\text{ext}dV \end{equation}
Si l'expansion est réversible, à chaque instant la pression intérieure est ajustée à la pression extérieure. Donc $P_\text{ext} = P_\text{int}$ à chaque instant :
\begin{equation} \delta W = -P_\text{int}dV = -NR_uT\frac{dV}{V} \end{equation}
Après intégration, on trouve l'équation \ref{eq:q2_8}. On observe donc que pour une détente isotherme : $W<0$, le travail est fourni par le système.
\fi

\subsection{Donner la définition de la pression partielle $p_i$ du constituant $i$
dans un mélange homogène de $n$ constituants.
Dériver l'expression pour la production d'entropie d'opération de mélange des
$n_A$ moles d'un gaz idéal $A$ avec $n_B$ moles d'un gaz idéal $B$.
Initialement, les deux gaz occupent des volumes différents $V_A$ et $V_B$
séparés par un diaphragme.}
La pression partielle d'un constituant $i$ est la pression qu'il aurait s'il occupait seul le volume total du mélange.

Avant mélange (état 1), les constituants occupent aux mêmes conditions de pression
et de température des volumes séparés.
L'entropie du système est alors la somme des entropies individuelles pondérées par les fractions
définissant le mélange, soit, en notant l'entropie molaire $S_m$:
\[ S_{m,1} = [A] S_{mA}(p,T) + [B] S_{mB}(p,T) \]
Après mélange (état 2), chaque constituant s'étant détendu de façon isotherme dans tout
le volume disponible, de la pression totale $p$ aux pressions partielles $p_A$ et $p_B$,
l'entropie prend pour valeur
\[ S_{m,2} = [A] S_{mA}(p_A,T) + [B] S_{mB}(p_B,T). \]
L'opération de mélange a donc entrainé l'accroissement d'entropie molaire:
\[ \Delta S_m = [A] \left(S_{mA}(p_A,T) - S_{mA}(p,T)\right) +
[B] \left(S_{mB}(p_B,T) - S_{mB}(p,T)\right) \]
Connaissant la variation d'entropie molaire pour une transformation isotherme
$ \Delta S_m = -\Ru\ln(p_2/p_1) $, on trouve
\[ \Delta S_m = -\Ru \left( [A]\ln\frac{p_A}{p} + [B]\ln\frac{p_B}{p} \right) \]

\iffalse
Considérons un espace fermé séparé en deux domaines différents $A$ et $B$, séparés par un diaphragme. Les volumes de ces deux domaines sont $V_A$ et $V_B$. Dans le domaine $A$ il y a $\eta_A$ moles du gaz $A$ et dans le domaine $B$ il y a $\eta_B$ moles du gaz $B$. Ce système est à l'équilibre et en équilibre avec l'extérieur. Sa pression à cet état d'équilibre est $p_0$ et sa température est $T_0$. Les entropies des deux gaz sont $S_{0,A}$ et $S_{0,B}$. De plus, on les considère comme les entropies de références pour les gaz. L'entropie totale du système est 
\begin{equation} S_0 = S_{0,A} + S_{0,B} \end{equation}
À l'instant où on enlève le diaphragme, les deux gaz se mélange. À la fin de ce processus on aura un mélange homogène à l'équilibre et en équilibre avec l'extérieur. Ceci signifie que la pression et la température à la fin de l'opération de mélange sont identiques :
\begin{equation} T = T_0 \qquad \text{et} \qquad p = p_0 \end{equation}
Néanmois, le volume occupé par chaque gaz est modifié. L'entropie du gaz $A$ au nouvel état d'équilibre est donnée par la relation :
\begin{align} dS &= c_p\frac{dT}{T} - R\frac{dp}{p} \\ S_A - S_{0,A} &= c_{p,A}\ln\left(\frac{T}{T_0}\right) - \eta_AR_u\ln\left(\frac{p_A}{p_0}\right) \\ &= -\eta_AR_u\ln\left(\frac{\eta_AR_uT}{V_A + V_B}\frac{V_A}{\eta_AR_uT_0}\right) \\ &= -\eta_AR_u\ln\left(\frac{V_A}{V_A+V_B}\right) \end{align}
Le volume $V_A$ est donné par l'équation d'état du gaz parfait :
\begin{equation} V_A = \frac{\eta_AR_uT_0}{p_0} \end{equation}
Le nouveau mélange homogène est aussi un gaz parfait dont le volume est donné par l'équation d'état du gaz parfait :
\begin{equation} V_A + V_B = \frac{(\eta_A+\eta_B)R_uT_0}{p_0} \end{equation}
On obtient donc 
\begin{align} S_A - S_{0,A} &= -\eta_AR_u\ln\left(\left[A\right]\right) \\ S_B - S_{0,B} &= -\eta_BR_u\ln\left(\left[B\right]\right)\end{align}
où $\left[i\right]$ est la fraction molaire du gaz $i$. La variation d'entropie lors du mélange est donc :
\begin{equation} S-S_0 = S_A - S_{0,A} + S_B - S_{0,B} = -R_u\left(\eta_A\ln\left(\left[A\right]\right) + \eta_B\ln\left(\left[B\right]\right)\right) \end{equation}
Pour obtenir l'expression de la production d'entropie spécifique, il faut encore diviser cette dernière relation par le nomre total de moles du mélange :
\begin{equation} s-s_0 =  -R_u\left(\left[A\right]\ln\left(\left[A\right]\right) + \left[B\right]\ln\left(\left[B\right]\right)\right) \end{equation}
\fi

\include{parts/03_transferts_chaleur}
\include{parts/04_machines_thermodynamiques}
\include{parts/05_thermodynamique_vapeurs}
\include{parts/06_machine_compression}
\include{parts/07_machine_detente}
\include{parts/08_air_humide}
\include{parts/09_turbines_gaz}
\include{parts/10_cycles_moteurs}
\include{parts/11_cycles_frigorifiques}


\end{document}
