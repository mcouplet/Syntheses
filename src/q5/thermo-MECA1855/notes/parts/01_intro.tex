\section{Introduction et rappels}
\subsection{Est-ce que la chaleur échangée entre un système et son extérieur est une variable d'état du système ? Justifier votre réponse.}
Une \emph{variable d'état} est un paramètre qui caractérise un état et non une évolution entre deux états.
La chaleur échangée $Q$ entre un système et son extérieur décrit le passage d'un état à un autre et ne peux pas être mesurée pour un état thermodynamique donné.
Dans le cas d'une transformation (1-2) sans dissipation par frottements, on a par le premier principe
\[ Q_{12} = U_2 - U_1 + \int_1^2 p \dif v. \]
Dans le diagramme $(p,v)$ de la transformation, on peut passer de l'état 1 à l'état 2 par n'importe quelle courbe.
La valeur de $\int_1^2 p \dif v$ dépend de la courbe qu'on prend, d'où on conclut que $Q_{12}$ ne dépend pas de l'état initial et final mais bien du chemin parcouru pour passer de l'état initial à l'état final.

\subsection{Donner la définition d'un processus \textit{adiabatique}. 
Donner un exemple d'un processus adiabatique réversible ainsi qu'un exemple d'un processus adiabatique irréversible.}
Un processus est dit \emph{adiabatique} si lors de ce processus le système n'échange pas de chaleur avec son environnement ($Q = 0$). 
Cela n'implique pas pour autant que la température du système reste constante. 
Par le premier principe, seul un travail mécanique peut encore modifier l'état du système.

Une transformation est dite réversible si, partant d'un état donné, on pourrait la concevoir comme pouvant se produire indifféremment dans un sens ou dans l'autre.
Le concept de réversibilité est idéal : en pratique, tout processus est irréversible.
L'expression énergétique du second principe de la thermodynamique, aussi appelé principe d'évolution, s'écrit 
\[ Q + W_\mathrm{f} = \int_1^2T \dif S \]
où $\dif S$ est la différentielle de la fonction d'état entropie et $W_\mathrm{f}$ le terme dissipatif. 
Dans la réalité ce terme ne sera jamais nul. 
\begin{itemize}
	\item Dans le cas d'un processus \textbf{réversible} on a $W_\mathrm{f} = 0$. 
		Un processus adiabatique réversible est donc aussi isentropique car $\dif S = \delta Q/T = 0$. 
		Un exemple d'un tel processus se trouve dans le cycle de Carnot qui comprend une détente ainsi qu'une compression adiabatique réversible. 
		Il est à noter que le cycle de Carnot est un cycle théorique.
	\item Dans le cas d'un processus \textbf{irréversible} on a $W_\mathrm{f} > 0$.
		On retrouve une détente adiabatique irréversible dans les cycles frigorifiques. 
\end{itemize}

\subsection{Donner la définition mathématique ainsi que la signification physique des chaleurs massiques $c_p$ et $c_v$.}
Physiquement, la chaleur massique est l'action calorifique nécessaire pour obtenir un accroissement de température d'un degré Kelvin d'un kilo d'une substance considérée dans des conditions bien déterminées.
Dans les transformations à pression ou volume constants, on note les chaleurs massiques $c_p$ et $c_v$, respectivement.
Mathématiquement, la définition s'écrit
\[ c_p \eqdef \lim_{\Delta T \to 0} \frac{Q_p}{\Delta T}, \qquad c_v \eqdef \lim_{\Delta T \to 0} \frac{Q_v}{\Delta T} \]
où $Q_p$ et $Q_v$ désignent les actions calorifiques respectives nécessaires au relèvement $\Delta T$ de température à pression et à volume constant.

Par la définition, on peut écrire
\begin{equation*}
	\delta Q_v = c_v \dif T, \qquad \delta Q_p = c_p \dif T. 
\end{equation*}
Le premier principe nous donne la relation suivante
\begin{equation*} 
	\dif U + p \dif v = \dif H - v \dif p = \delta(Q + W_\mathrm{f}).
\end{equation*}
Dans le cas d'une transformation \emph{sans dissipation par frottements ($W_\mathrm{f} = 0$)} à volume constant, on a donc
\[ \dif U = c_v \dif T, \qquad c_v = \thermodiff{U}{T}{v} \]
et à pression constante,
\[ \dif H = c_p \dif T, \qquad c_p = \thermodiff{H}{T}{p} \]

\subsection{Donner la définition de l'énergie libre $F$ de Helmholtz ainsi que de l'enthalpie libre $G$ de Gibbs. 
Écrire l'équation de Gibbs sous la forme de différentielle de $F$ ainsi que de $G$.}
L'énergie libre de Helmholtz $F$ et l'enthalpie libre de Gibbs $G$ sont définies par
\begin{align*}
	F &\eqdef U - TS \\
	G &\eqdef H - TS.
\end{align*}
On trouve immédiatement leurs différentielles :
\begin{align*}
	\dif F &= \dif U - T \dif S - S \dif T \\
	\dif G &= \dif H - T \dif S - S \dif T.
\end{align*}
L'équation de Gibbs est la relation
\[ T \dif S = \dif U + p \dif v = \dif H - v \dif p \]
qui permet d'obtenir
\begin{align*}
	\dif F &= - p \dif v - S \dif T \\
	\dif G &= v \dif p - S \dif T.
\end{align*}

\iffalse % Interesting, but does not answer the question
\paragraph{L'énergie libre $F$ de Helmholtz} est une fonction d'état extensive dont la variation permet d'obtenir le travail utile susceptible d'être fourni par un système fermé, à température constante, au cours d'une transformation réversible. 
Considérons un système thermodynamique (fermé) évoluant d'un état 1 à un état 2, transformation que l'on suppose totalement réversible ($W_f = 0$) et à température constante $T$. Le premier principe de la thermodynamique s'exprime par :
\begin{equation} \Delta U + \Delta K + g\Delta z = Q + \underline{W_e} \end{equation}
où $\Delta U$ désigne la variation de l'énergie interne, $\Delta K$ la variation de l'énergie cinétique, $g\Delta z$ la variation de l'énergie potentielle, $Q$ l'échange de chaleur et $\underline{W_e}$ le travail des forces extérieures. On introduit $W = -W_e$ qui correspond au travail récupérable par le milieu extérieur et on suppose $\Delta K = g\Delta z = 0$. On obtient alors
\begin{equation} W = Q - \Delta U \label{eq:Q1ePrincipe}\end{equation}
Le 2\ieme principe de la thermodynamique s'exprime par:
\begin{equation} \Delta S = \frac{Q}{T} + W_f \qquad\Leftrightarrow\qquad Q = T\Delta S - TW_f \label{eq:Q2ePrincipe}\end{equation}
où $\Delta S$ est la variation d'entropie du système. Si on remplace \ref{eq:Q2ePrincipe} dans \ref{eq:Q1ePrincipe}, nous trouvons
\begin{equation} W = T\Delta S - TW_f - \Delta U = T(S_2 - S_1) - TW_f + U_1 - U_2 \end{equation}
Il est évident, d'après la relation obtenue, que $W$ sera maximum si la transformation est totalement réversible ($W_f = 0$) :
\begin{equation} W_\text{max} = (U_1 - TS_1)-(U_2-TS_2)\end{equation}
On définit alors 
\begin{equation} F = U - TS \end{equation} 
Ce qui nous donne au final 
\begin{equation} W_\text{max} = -\Delta F \qquad\text{où}\qquad \underbrace{\Delta F}_\text{travail utile} = \underbrace{\Delta U}_\text{énergie totale} - \underbrace{T\Delta S}_\text{énergie inutilisable}\end{equation}

\paragraph{Forme différentielle de $F$}
\begin{equation} \left.\begin{array}{ll} dF &= d(U-TS) = dU - TdS - SdT \\ dU &= -pdV + TdS \end{array}\right\} \Rightarrow dF = -pdV - SdT\end{equation}

\paragraph{L'enthalpie libre $G$ de Gibbs} est une fonction d'état extensive dont la variation permet d'obtenir l'enthalpie utile susceptible d'être fournie par un système fermé, à température et à pression constante, au cours d'une transformation réversible. On considère le même système que précedemment avec la condition supplémentaire que la pression est constante. Par la définition de l'enthalpie :
\begin{equation} H \triangleq U + pV\end{equation}
De la même manière que pour l'énergie libre de Helmholtz, on trouve :
\begin{equation} W = Q - \Delta H \label{H1ePrincipe}\end{equation}
où $W$ correspond à nouveau au travail récupérable par le milieu extérieur. Si on remplace \ref{eq:Q2ePrincipe} dans \ref{H1ePrincipe} et qu'on prend on compte le fait que le processus est réversible, on obtient :
\begin{equation} W_\text{max} = (H_1 - TS_1) - (H_2 - TS_2) \end{equation}
On définit alors
\begin{equation} G = H - TS \end{equation}

\paragraph{Forme différentielle de G}
\begin{equation} \left.\begin{array}{ll} dG &= d(H-TS) = dH - TdS - SdT \\ dH &= Vdp + TdS \end{array}\right\} \Rightarrow dG = Vdp - SdT \label{eq:gibbs-duhem}\end{equation}

\fi

\subsection{Dériver l'équation $\alpha = p\beta \kappa$.}
Les variables d'état $p$, $v$ et $T$ d'un fluide sont liées par une équation d'état, que l'on peut écrire sous forme du système différentiel :
\[
	\begin{pmatrix} \dif p \\ \dif v \\ \dif T \end{pmatrix} = 
	\begin{pmatrix}
		0 & \thermodiff{p}{v}{T} & \thermodiff{p}{T}{v} \\
		\thermodiff{v}{p}{T} & 0 & \thermodiff{v}{T}{p} \\
		\thermodiff{T}{p}{v} & \thermodiff{T}{v}{p} & 0
	\end{pmatrix}
	\begin{pmatrix} \dif p \\ \dif v \\ \dif T \end{pmatrix} 
\]
On définit les coefficients suivants :
\begin{itemize}
	\item le coefficient de dilatation isobare
		$ \alpha \eqdef \frac{1}{v} \thermodiff{v}{T}{p} $ ;
	\item le coefficient de dilatation isochore
		$ \beta \eqdef \frac{1}{p} \thermodiff{p}{T}{v} $ ;
	\item le coefficient de compressibilité isotherme
		$ \kappa \eqdef -\frac{1}{v} \thermodiff{v}{p}{T} $.
\end{itemize}
Grâce à ces coefficients, on peut réécrire le système :
\[
	\begin{pmatrix} \dif p \\ \dif v \\ \dif T \end{pmatrix} = 
	\begin{pmatrix}
		0 & -\frac{1}{\kappa v} & \beta p \\
		-\kappa v & 0 & \alpha v \\
		\frac{1}{\beta p} & \frac{1}{\alpha v} & 0
	\end{pmatrix}
	\begin{pmatrix} \dif p \\ \dif v \\ \dif T \end{pmatrix} 
\]
On trouve l'équation recherchée à l'aide des variations de deux des trois variables d'état: 
\begin{align*}
	\dif p &= -\frac{1}{\kappa v} \dif v + \beta p \dif T \\
	\dif v &= -\kappa v \dif p + \alpha v \dif T
\end{align*}
En remplaçant l'expression de $\dif v$ dans l'expression de $\dif p$, on obtient
\[ \dif p = \dif p - \frac{\alpha}{\kappa} \dif T + \beta p \dif T \]
d'où on tire $ \alpha = p \beta \kappa $.
L'équation d'état d'une substance est donc complètement déterminée par la connaissance de deux des trois coefficients $\alpha$, $\beta$ ou $\kappa$.

\subsection{Démontrer que les relations suivantes sont valables pour toutes les espèces.}
\[ \thermodiff{T}{p}{S} = \thermodiff{v}{S}{p}, \qquad
	\thermodiff{S}{p}{T} = -\thermodiff{v}{T}{p} \]

On part de l'équation de Gibbs sous la forme de différentielle de $H$ et $G$ :
\begin{align*}
	\dif H &= T \dif S + v \dif p \\
	\dif G &= v \dif p - S \dif T
\end{align*}
qui nous donne les dérivées partielles suivantes
\[ \begin{matrix}
	\thermodiff{H}{S}{p} = T & \thermodiff{H}{p}{S} = v \\
	\thermodiff{G}{p}{T} = v & \thermodiff{G}{T}{p} = -S
\end{matrix} \]
Par le théorème de Schwarz
\footnote{Si $f$ est deux fois dérivable en un point, alors sa matrice hessienne y est symétrique}
appliqué aux dérivées partielles de $H$ et $G$ :
\begin{align*}
	\pdv{H}{S}{p} &= \pdv{H}{p}{S} \\
	\pdv{G}{T}{p} &= \pdv{G}{p}{T}
\end{align*}
on trouve les relations recherchées :
\begin{align}
	%\thermodiff{T}{v}{S} &= -\thermodiff{p}{S}{v} \\
	\thermodiff{T}{p}{S} &= \thermodiff{v}{S}{p} \notag \\
	\thermodiff{v}{T}{p} &= -\thermodiff{S}{p}{T}.
\end{align}

\subsection{Dériver l'équation $c_p-c_v = \alpha\beta pvT$.\label{q:1_7}}
On part de l'équation de Gibbs
\[ T \dif S = \dif U + p \dif v \]
La différentielle de $U$ peut s'écrire de trois manières différentes :
\[
	\begin{pmatrix} \dif U \\ \dif U \\ \dif U \end{pmatrix} = 
	\begin{pmatrix}
		0 & \thermodiff{U}{v}{T} & \thermodiff{U}{T}{v} \\
		\thermodiff{U}{p}{T} & 0 & \thermodiff{U}{T}{p} \\
		\thermodiff{U}{p}{v} & \thermodiff{U}{v}{p} & 0
	\end{pmatrix}
	\begin{pmatrix} \dif p \\ \dif v \\ \dif T \end{pmatrix}
\]
On peut donc exprimer $T\dif S$ de trois manières différentes :
\[ 
	\begin{pmatrix} T \dif S \\ T \dif S \\ T \dif S \end{pmatrix} = 
	\begin{pmatrix}
		0 & \thermodiff{U}{v}{T} + p & \thermodiff{U}{T}{v} \\
		\thermodiff{U}{p}{T} & p & \thermodiff{U}{T}{p} \\
		\thermodiff{U}{p}{v} & \thermodiff{U}{v}{p} + p & 0
	\end{pmatrix}
	\begin{pmatrix} \dif p \\ \dif v \\ \dif T \end{pmatrix}
\]
On annule le second terme de la diagonale avec $ \dif v = \thermodiff{v}{p}{T} \dif p + \thermodiff{v}{T}{p} \dif T $ :
\[ 
	\begin{pmatrix} T \dif S \\ T \dif S \\ T \dif S \end{pmatrix} = 
	\begin{pmatrix}
		0 & \thermodiff{U}{v}{T} + p & \thermodiff{U}{T}{v} \\
		\thermodiff{U}{p}{T} + p \thermodiff{v}{p}{T} & 0 & \thermodiff{U}{T}{p} + p \thermodiff{v}{T}{p} \\
		\thermodiff{U}{p}{v} & \thermodiff{U}{v}{p} + p & 0
	\end{pmatrix}
	\begin{pmatrix} \dif p \\ \dif v \\ \dif T \end{pmatrix}
\]
Par introduction de nouveaux coefficients calorifiques, on note
\[ 
	\begin{pmatrix} T \dif S \\ T \dif S \\ T \dif S \end{pmatrix} = 
	\begin{pmatrix}
		0	& l_T	& c_v	\\
		h_T	& 0		& c_p	\\
		h_v	& l_p	& 0
	\end{pmatrix}
	\begin{pmatrix} \dif p \\ \dif v \\ \dif T \end{pmatrix}
\]
où on définit ces coefficients :
\begin{itemize}
	\item coefficient calorifique de dilatation à température constante :
		\[ l_T = \thermodiff{U}{v}{T} + p = \thermodiff{H}{v}{T} - v \thermodiff{p}{v}{T} \]
	\item coefficient calorifique de compressbilité à température constante :
		\[ h_T = \thermodiff{U}{p}{T} + p \thermodiff{v}{p}{T} = \thermodiff{H}{p}{T} - v \]
	\item coefficient calorifique de compressibilité sous volume constant :
		\[ h_v = \thermodiff{U}{p}{v} = \thermodiff{H}{p}{v} - v \]
	\item coefficient calorifique de dilatation sous pression constante :
		\[ l_p = \thermodiff{U}{v}{p} + p = \thermodiff{H}{v}{p} \]
\end{itemize}
On annule la dernière colonne avec $ \dif T = \thermodiff{T}{p}{v} \dif p + \thermodiff{T}{v}{p} \dif v = \frac{1}{\beta p} \dif p + \frac{1}{\alpha v} \dif T $ :
\[ 
	\begin{pmatrix} T \dif S \\ T \dif S \\ T \dif S \end{pmatrix} = 
	\begin{pmatrix}
		\frac{c_v}{\beta p}			& l_T + \frac{c_v}{\alpha v}	& 0	\\
		h_T + \frac{c_p}{\beta p}	& \frac{c_p}{\alpha v}			& 0	\\
		h_v							& l_p							& 0
	\end{pmatrix}
	\begin{pmatrix} \dif p \\ \dif v \\ \dif T \end{pmatrix}
\]
Par identification des coefficients, on peut exprimer les coefficients $l_T$, $h_T$, $h_v$ et $l_p$ en fonction des chaleurs massiques $c_p$ et $c_v$ et des coefficients de dilatation $\alpha$ et $\beta$ :
\[
	l_T = \frac{c_p-c_v}{\alpha v} \qquad
	h_T = \frac{c_v-c_p}{\beta p} \qquad
	h_v = \frac{c_v}{\beta p} \qquad
	l_p = \frac{c_p}{\alpha v}
\]
On peut alors exprimer les dérivées partielles de l'entropie en fonction de $\alpha$, $\beta$, $c_p$ et $c_v$ uniquement :
\[
	\begin{pmatrix}
		0						& \thermodiff{S}{v}{T}	& \thermodiff{S}{T}{v}	\\
		\thermodiff{S}{p}{T}	& 0						& \thermodiff{S}{T}{p}	\\
		\thermodiff{S}{p}{v}	& \thermodiff{S}{v}{p}	& 0
	\end{pmatrix}
	=
	\begin{pmatrix}
		0							& \frac{c_p-c_v}{\alpha v T}	& \frac{c_v}{T}	\\
		\frac{c_v-c_p}{\beta p T}	& 0								& \frac{c_p}{T}	\\
		\frac{c_v}{\beta p T}		& \frac{c_p}{\alpha v T}		& 0
	\end{pmatrix}
\]
Le théorème de Schwarz appliqué aux dérivées partielles de $G$ nous donne la relation
\[ \thermodiff{v}{T}{p} = - \thermodiff{S}{p}{T} \]
En exprimant ces différentielles par les coefficient calorifiques, on trouve
\begin{align*}
	\alpha v &= \frac{c_p - c_v}{\beta p T} \\
	c_p - c_v &= \alpha \beta p v T
\end{align*}
Il résulte de ces considérations que les fonctions d'état peuvent se calculer si on connait deux des trois coefficients physiques $\alpha$, $\beta$, $\kappa$ et la chaleur massique $c_p$ (ou $c_v$).

\subsection{Dériver l'équation $l_T = \frac{c_p-c_v}{\alpha v}$.}
Le développement pour cette question est un sous-ensemble de la réponse à la question précédente.

